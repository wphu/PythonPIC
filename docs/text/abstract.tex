Algorytmy Particle-in-Cell ("cząstka w komórce") to jedne z najbardziej zbliżonych do fundamentalnej fizyki metod symulacji
materii w stanie plazmy. Zastosowany w nich lagranżowski opis cząsteczek pozwala na dokładne odwzorowanie dynamiki ruchu
elektronów i jonów. Jednocześnie, ewolucja pola elektromagnetycznego na Eulerowskiej siatce dokonywana zamiast bezpośredniego obliczania
oddziaływań międzycząsteczkowych pozwala na znaczące przyspieszenie etapu obliczenia oddziaływań międzycząsteczkowych.

Python jest wysokopoziomowym, interpretowanym językiem programowania, którego atutami są szybkie prototypowanie,

Python znajduje zastosowania w analizie danych, uczeniu maszynowym (zwłaszcza w astronomii). W zakresie symulacji
w ostatnich czasach powstały kody skalujące się nawet w zakres superkomputerów, np. w mechanice płynów % TODO: PyFR

Atutem Pythona w wysokowydajnych obliczeniach jest łatwość wywoływania w nim zewnętrznych bibliotek napisanych na przykład w C lub Fortranie, co pozwala
na osiągnięcie podobnych rezultatów wydajnościowych jak dla kodów napisanych w C.

Innym podejściem do optymalizacji kodu napisanego w Pythonie jest kompilacja Just-In-Time
